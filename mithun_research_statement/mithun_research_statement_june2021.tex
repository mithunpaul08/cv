\def\pilot{{\tt PILOT}}
\def\RR{{\tt SAIL}}
\def\bb{\noindent\hspace{.46cm}$\bullet$\hspace{.19cm}}




\documentclass[10pt]{article}
\usepackage{graphicx}
\usepackage{wrapfig}
%\usepackage[most]{tcolorbox}


%\usepackage{simplemargins}
%\setallmargins{1in}

\usepackage{url}
\evensidemargin -.2cm
\oddsidemargin -.2cm
\setlength\topmargin{-0.7in}
\setlength{\textwidth}{16cm}   % width of body
\setlength{\textheight}{9.2in}   % heigth of body
%\setlength{\sectionskip}{2ex}
\setlength{\itemsep}{-1in}
\setlength{\leftmarginii}{.05in}

\newcommand{\ve}[1]{{\em #1}} % venue of publication
\newcommand{\ti}[1]{``#1''} % title of publication
\newcommand{\lmt}{(limited to period starting 01/2013)}

\newcommand{\todo}[1]{\textcolor{red}{TODO: #1}}

\begin{document}


%\pagestyle{empty}

\begin{center}
{\Large \bf Research Statement\\
\smallskip
{\large \bf Mitch Paul Mithun}}
\end{center}


\section{Research Accomplishments}

My research is on natural language processing, that is, getting computers to understand human language. Specifically, I work on  building natural language inference (NLI) models which can be used for identifying the trustworthiness of online information. This is an important challenge: around 67\% Americans reported that fake news causes confusion about simple information that they read online and 54\% say that fake news affects their confidence in other people they interact with (Pew Research, 2019), making fake news a very important societal issue.

While there have been several methods created to address and automate online fact verification, humans still seem to have a better ability than machines to distinguish good facts from bad facts. I believe this is because we use background and domain knowledge while these methods only look at facts as independent silos. Inspired from this observation, my research aims on building robust methods for fact verification by aggregating information across multiple facts (e.g., based on whether the facts support or refute with each other) and aim to reach global conclusions from this aggregation. 
To achieve information aggregation for fact verification, relevant facts must be mined and included into a graph where nodes represent facts and edges represent the relation between them. 
Further, these graphs must be analyzed using a holistic scoring and ranking algorithm which assigns a truthfulness score for any given piece of text.

As a first step towards this goal, I believe, it is important that these graphs be made robust by creating neural network methods which can learn knowledge that is domain transferable. To this end, we\footnote{Unless otherwise specified, plural first-person pronouns refer to my collaborators and I, where collaborators come mostly from the computational language understanding (CLU) lab at the University of Arizona.}  have developed two important research techniques, data distillation and model distillation. We showed that these techniques enable the neural network models to learn the underlying semantics which enable them to perform well on other datasets and domains, other than the ones they were trained on.
I detail these techniques below:
\subsection{Neural networks that can learn domain transferable information}While neural networks produce state-of-the-art performance in many NLP tasks, it has been shown that several such models depend heavily on certain statistical nuances and lexical artifacts found in these datasets, information that transfers poorly between domains.
For example when we inspected the  the attention weights assigned by the model trained in a fact verification task, we found that attention weights tend to be directed towards words and noun phrases that are more likely to be domain specific, which considerably
impacts performance in an out-of-domain setting.

Hence I felt it is important to first device techniques that can enable models to learn the underlying semantics and mitigate the dependence on lexicalized information.
In my work, I have approached the goal of building domain transferable neural networks from multiple perspectives:
\subsubsection{Data Distillation}
To mitigate this dependence on lexicalized information 
we devised some novel de-lexicalization methods
that replace concepts with their respective semantic classes \cite{panenghat2020towards,suntwal-etal-2019-importance}. For example one type of delexicalization included replacing the proper noun `J.R.R.Tolkien' with its named entity class, \texttt{PERSON}. In another technique we replace `J.R.R.Tolkien' with a more finer semantic class, \texttt{AUTHOR}. Further we ensure that such replacements done in both claims and evidence data in a fact verification task are aligned.  For instance, if a concept appears first in the claim, it is assigned an identifier post-fixed with C{\em n}; if it appears only in evidence then it is post-fixed with E{\em n}, where C indicates that the entity appeared first in the claim, E indicates that the entity first appeared in the evidence, and {\em n} indicates the {\em n}th observed entity. By generalizing these concepts to more abstract forms, we demonstrated that these methods encourage neural networks to look beyond lexical patterns and extract underlying semantic knowledge that can be transferred from one domain to another. For example a model trained on our approach outperformed a model trained on the original, fully-lexicalized texts by over 10\% accuracy when evaluated out of domain. For example a model trained on our approach outperformed a model trained on the original, fully-lexicalized texts by over 10\% accuracy when evaluated out of domain.
 
 
\subsubsection{Model Distillation} 



 A critical unsolved issue that remained after data distillation was \textit{how much} delexicalization should be applied?
 While a little delexicalization (e.g., `J.R.R.Tolkien' to \texttt{PERSON}) helps reduce over-fitting, too much delexicalization (e.g., `J.R.R.Tolkien' to \texttt{AUTHOR}) might discard useful information. To solve this problem we introduced Group Learning (GL), a \textit{novel model distillation architecture} based on the
 teacher-student paradigm \cite{hinton2015distilling}.
 Specifically, in our architecture multiple models have access to different perspectives of the same data, each delexicalized with a different level of granularity. These models then compete with each other, and also learn from each other through pair-wise consistency losses, to arrive at the right level of abstraction needed for data distillation. We showed that not only is our approach classifier agnostic, it also transfers learning across various domains \cite{mithun2021data}. Further, this approach thus reduces the computational overhead (and, thus, carbon footprint) caused by having to train neural network based methods every time they need to make predictions on new fact verification domains or datasets.
 
\subsection{Other Research Contributions }

Last but not the least, along with my research on fact verification, I have been involved with many other collaborative research and development efforts in our lab. For example I contributed to the DARPA World Modelers project, which resulted in the creation of Eidos, an open-domain
machine reading system designed to extract
causal relations from natural language especially in documents relevant to national and global security. Specifically, I contributed to the creation of a method for quickly and efficiently generating groundings for a set of gradable adjectives, i.e., adjectives that can take a range of magnitudes (e.g., \textit{small}) \cite{sharp2018grounding}. The resulting trained model was able to estimate the impact of novel instances of gradable adjectives found in text. This is an important contribution because it reduced the ambiguity in the machine reading process of texts which describe causal interactions using vague, high-level language. 

Currently I work on DARPA Habitus project, an effort to create methodologies for creating predictive, causal models of local systems based on input from local population in African Countries. For example, we have devised models that can predict the credit worthiness of a farmer based on his answers to questions in certain agricultural surveys. This serves as input to creating
comprehensive models of local systems which would be generalizable across regions and populations. This is important because it makes local knowledge available to military operators, providing them with an insider view to support decision making.
\section*{Future Work}
To date, my work has led to publications in several top NLP conferences (EMNLP, NAACL, and ACL workshop). Looking forward, I will:
\begin{itemize}
\item Address grand challenges with a tangible impact in our society. For example, I plan to utilize the fact verification algorithms I have created to compare any given text (e.g., a news article) with other similar texts, aggregate this global information into a graph form, and recursively traverse this graph to arrive at a \textit{veracity} score for each text. By automatically calculating, associating and displaying such a score for articles found on the social media, I hope to enable the population of interest to assess the truthfulness of such articles.

\item Continue to work on fundamental research issues. For example, I would like to use my NLI techniques I towards building a universal natural language processing model with few shot entailment (few shot is a methodology where we train a model in a source domain and use it in target domain without re-training it. This is done when we cannot obtain rich annotation on a target domain or task). 
Specifically, this is inspired from \cite{yin2020universal} and I am currently involved in a collaborative effort with the first author of this work, where we are trying to understand how well does a pre-trained textual entailment system generalize across domains with only a handful of domain specific examples. Next we plan to understand when is it worth transforming an NLP task into a textual entailment problem. For example a question answering problem can be formulated as a textual entailment problem where the document acts as the premise, and the question, answer
candidate pair, after converting into a natural sentence,
acts as the hypothesis. We hope to demonstrate that our framework enables a pre-trained entailment model to work well on new entailment domains in a few-shot setting. Further we hope to show its effectiveness as a unified solver for several downstream NLP tasks such as question answering  when the end-task annotations are limited.

\end{itemize}

{\flushleft Signature:}
\bigskip
\bigskip
\bibliographystyle{acm}
\bibliography{custom}
\end{document}




