\def\pilot{{\tt PILOT}}
\def\RR{{\tt SAIL}}
\def\bb{\noindent\hspace{.46cm}$\bullet$\hspace{.19cm}}




\documentclass[10pt]{article}

\usepackage{graphicx}
\usepackage{wrapfig}
\usepackage[most]{tcolorbox}


%\usepackage{simplemargins}
%\setallmargins{1in}

\usepackage{url}
\evensidemargin -.2cm
\oddsidemargin -.2cm
\setlength\topmargin{-0.7in}
\setlength{\textwidth}{16cm}   % width of body
\setlength{\textheight}{9.2in}   % heigth of body
%\setlength{\sectionskip}{2ex}
\setlength{\itemsep}{-1in}
\setlength{\leftmarginii}{.05in}

\newcommand{\ve}[1]{{\em #1}} % venue of publication
\newcommand{\ti}[1]{``#1''} % title of publication
\newcommand{\lmt}{(limited to period starting 01/2013)}

\newcommand{\todo}[1]{\textcolor{red}{TODO: #1}}

\begin{document}


%\pagestyle{empty}

\begin{center}
{\Large \bf Research Statement\\
\smallskip
{\large \bf Mitch Paul Mithun }}
\end{center}
%\smallskip

\subsection*{Contributions to Course Development}

When I started at UA there were limited opportunities for students interested in NLP. The only NLP course offered at the time was Statistical Natural Language Processing, in Linguistics. While this is a great course, it is not sufficient to cover the entire field of NLP. 
%In particular, this course covers only some of the machine learning (ML) models used in NLP, and does not address at all the building of end-to-end NLP applications. 
To address these curriculum limitations, I created two courses in my first two years at UA: ISTA 456/556 Text Retrieval and Web Search, which focuses on information retrieval (this course has been renumbered as CSC 483/583), and ISTA 455/555 Applied Natural Language Processing, which is designed as a continuation of the Statistical NLP course, emphasizing the development of complete, end-to-end NLP applications.  Additionally, I contributed to the development of ISTA 116, a foundational course that is fundamental to students interested in NLP.

More recently, I contributed to the creation of a single landing page for NLP at UA (\url{http://nlp.arizona.edu}), so students have an immediate understanding of the availability of NLP faculty on campus. The next step is to create a unified NLP curriculum. For this, I will work together with the Human Language Technology program in Linguistics\footnote{I currently hold a courtesy appointment in Linguistics.} and the other faculty listed in the website above to make sure that the courses we design and teach complement well the courses currently offered. For example, I have recently designed a course called Algorithms for Natural Language Processing, which will cover more modern algorithms for NLP that are not covered in the Statistical Natural Language Processing course such as recurrent neural networks.

\subsection*{Teaching Principles}

As a teacher, I explore new ways of presenting the material in order to emphasize the big picture behind the material covered. Showing how the fundamental principles covered in the subject matter impact real life (e.g., smarter search engines due to natural language understanding; novel disease treatments due to big data analysis) engages the students, and gives them a glimpse of where they could potentially use the material in a real-world career.

A second focus of my teaching work is personalization of information dissemination. 
I have personalized the learning experience in my courses or laboratories through customizable assignments and projects. This allows students to choose a preferred learning path that emphasizes either practice, or theory, or both.
I also believe that teaching is not a one-way process. I prefer to receive continuous feedback from students (either through comments during class or a series of ungraded quizzes) about the quality of my teaching and of the course materials. In the past few years I have a included an informal teacher and course evaluation (TCE) quiz that students take mid semester. This helps me adjust the delivery of the course to the students currently taking it. This is important especially for NLP courses, which tend to be attended by students from multiple departments, with various backgrounds. 

Furthermore, I aim to further improve engagement through active learning. My preferred method is through course projects that tackle important issues. For example, the course project for the Text Retrieval and Web Search course in Spring 2017 is the Fake News Challenge (\url{http://www.fakenewschallenge.org}). We approach these projects top down, e.g., Why is the detection of fake news important?; How would you tackle it intuitively?; What knowledge do we need to actually implement these ideas? I have found this strategy to engage students considerably more than passively teaching about NLP and ML. 

Lastly, I try to be as hands-on as possible in my teaching activities. I regularly help students with their code. I continue to code, and continue to be one of the main contributors to our public software projects, which helps me be up to date on software development, and provide relevant practical feedback. I encourage students to approach me with any issues, ranging from theory to coding questions: I have an open-door policy in addition to regular weekly office hours. 


As a result of applying these teaching principles, I have received consistently high TCEs in my teaching activity at UA, with a score usually over 4.5/5 on my teaching effectiveness, the rating of the course, and usefulness of lecture and other materials. 


\end{document}




